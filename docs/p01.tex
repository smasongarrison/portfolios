% Options for packages loaded elsewhere
\PassOptionsToPackage{unicode}{hyperref}
\PassOptionsToPackage{hyphens}{url}
%
\documentclass[
]{article}
\usepackage{amsmath,amssymb}
\usepackage{iftex}
\ifPDFTeX
  \usepackage[T1]{fontenc}
  \usepackage[utf8]{inputenc}
  \usepackage{textcomp} % provide euro and other symbols
\else % if luatex or xetex
  \usepackage{unicode-math} % this also loads fontspec
  \defaultfontfeatures{Scale=MatchLowercase}
  \defaultfontfeatures[\rmfamily]{Ligatures=TeX,Scale=1}
\fi
\usepackage{lmodern}
\ifPDFTeX\else
  % xetex/luatex font selection
\fi
% Use upquote if available, for straight quotes in verbatim environments
\IfFileExists{upquote.sty}{\usepackage{upquote}}{}
\IfFileExists{microtype.sty}{% use microtype if available
  \usepackage[]{microtype}
  \UseMicrotypeSet[protrusion]{basicmath} % disable protrusion for tt fonts
}{}
\makeatletter
\@ifundefined{KOMAClassName}{% if non-KOMA class
  \IfFileExists{parskip.sty}{%
    \usepackage{parskip}
  }{% else
    \setlength{\parindent}{0pt}
    \setlength{\parskip}{6pt plus 2pt minus 1pt}}
}{% if KOMA class
  \KOMAoptions{parskip=half}}
\makeatother
\usepackage{xcolor}
\usepackage[margin=1in]{geometry}
\usepackage{graphicx}
\makeatletter
\def\maxwidth{\ifdim\Gin@nat@width>\linewidth\linewidth\else\Gin@nat@width\fi}
\def\maxheight{\ifdim\Gin@nat@height>\textheight\textheight\else\Gin@nat@height\fi}
\makeatother
% Scale images if necessary, so that they will not overflow the page
% margins by default, and it is still possible to overwrite the defaults
% using explicit options in \includegraphics[width, height, ...]{}
\setkeys{Gin}{width=\maxwidth,height=\maxheight,keepaspectratio}
% Set default figure placement to htbp
\makeatletter
\def\fps@figure{htbp}
\makeatother
\setlength{\emergencystretch}{3em} % prevent overfull lines
\providecommand{\tightlist}{%
  \setlength{\itemsep}{0pt}\setlength{\parskip}{0pt}}
\setcounter{secnumdepth}{-\maxdimen} % remove section numbering
\ifLuaTeX
  \usepackage{selnolig}  % disable illegal ligatures
\fi
\IfFileExists{bookmark.sty}{\usepackage{bookmark}}{\usepackage{hyperref}}
\IfFileExists{xurl.sty}{\usepackage{xurl}}{} % add URL line breaks if available
\urlstyle{same}
\hypersetup{
  pdftitle={Portfolio 1},
  hidelinks,
  pdfcreator={LaTeX via pandoc}}

\title{Portfolio 1}
\author{}
\date{\vspace{-2.5em}}

\begin{document}
\maketitle

How To's for Mason's Data Science Class

Aims of this document

Throughout her modules, Mason delves into detail on a number of
different elements, explaining their different uses and how they can be
useful to you. Similarly, when discussing how to do different tasks, she
will often provide different ways that aim can be accomplished. This
approach is very reasonable and is the one I would take -- it provides
training that goes above and beyond the specifics for this individual
class. Other web-based videos take this approach too.

One limitation of this approach, however, is it can be challenging to
accomplish fairly straightforward tasks, especially at the beginning of
the semester, with all of the information that is provided. Thus I put
together (for myself) this list of how to accomplish certain tasks, with
a particular emphasis on those related to Git / Github, though with a
brief description of some R studio tasks as well. This does not delve
into the `why's (e.g., it doesn't discuss the difference between Git and
Github) and is intended as a companion to what is already provided in
Mason's modules, as a brief reference guide. I'm making this publicly
available in case it's helpful to others. It is intended to be a `living
document,' to be updated to include more up-to-date information, other
useful tips, etc.

As a secondary aim, in some cases I have also provided descriptions of
how this approach differs from what was seen in my graduate class using
SPSS.

This document assumes that you have already: (1) installed R and R
studio, (2) set up a Github account, and (3) installed Github Desktop.

How do I find the files for a particular lab?

All the relevant files are stored in Github, in a ``repository''
associated with that lab. Mason's Github account is:
\url{https://github.com/DataScience4Psych} (not to be confused with
\url{https://datascience4psych.github.io/DataScience4Psych/}, as I did
originally). Go to her Github account, and under repositories, search
for the relevant one (e.g., by typing in lab 01). Then click on the
relevant repository.

Once you've done that, the relevant repository should show up. For
example, for the first lab, it should direct you to
\url{https://github.com/DataScience4Psych/lab-01-hello-r} . This
contains all the relevant files for the first lab. It looks like:

How do I get the relevant files on my computer?

You need to get the relevant files both on your computer, and on your
personal Github account. Thankfully, there's a way to do all of that in
two steps. The basic idea is you're going to just copy (`clone')
everything from Mason's account onto yours.

First you will get her files onto your Github account. To do that,
access the relevant repository from her Github account and:

\begin{enumerate}
\def\labelenumi{\arabic{enumi}.}
\tightlist
\item
  Click Use this template -\textgreater{} create a new repository
\item
  This will take you to a page where you will need to specify the owner
  (should be filled in already -- this is you), and a name for the lab
  that you'll fill in (e.g., lab 1).
\item
  You can click public or private (Mason's module explains the
  difference).
\item
  Click Create Repository
\end{enumerate}

This will then create a copy (clone) of Mason's repository on your
Github account. However, you still need to get the files onto Github
Desktop and onto your personal computer. To do that, from your
repository (that you just created):

\begin{enumerate}
\def\labelenumi{\arabic{enumi}.}
\tightlist
\item
  Click Code -\textgreater{} Open with Github Desktop
\item
  This will produce a pop-up with the address of your github, and the
  path on your computer. For example, for the 5th lab, mine looked like
  the following:
\end{enumerate}

\begin{enumerate}
\def\labelenumi{\arabic{enumi}.}
\setcounter{enumi}{2}
\tightlist
\item
  Typically this should all be correct. In that case, just click Clone.
\end{enumerate}

Notice that a folder with the relevant files has now been created on
your computer, and you can access the folder from within Github Desktop.

How do I find the relevant repository on Github Desktop?

If the relevant repository does not show up, then on the upper left, go
to Current Repository. You can change that to whatever repository you
are interested in.

How do I get the relevant files into Rstudio so that I can work on them?

\begin{enumerate}
\def\labelenumi{\arabic{enumi}.}
\item
  Within the relevant repository / folder, click on the .Rproj file.
  .Rproj stands for the entire project related to that assignment. It
  will pull up all the files you need within Rstudio.
\item
  Then, on the files pane, click on the .rmd file to pull up the R
  markdown file.
\end{enumerate}

How (Where) to answer the questions that Mason asks in her assignments?

Short Version: You will provide your answers in the R Markdown document.

Longer Version: I initially thought of an R markdown file as analagous
to an SPSS syntax file. That is accurate in the sense that the R
markdown file runs code sequentially, but it also can be considered as
analagous to a word document that you can embed R code in. One nice
element of this is that you can include both your code and your answers
to questions within the Markdown document, unlike my class, where you
had to write up the answers in a separate word document or by hand.

\end{document}
